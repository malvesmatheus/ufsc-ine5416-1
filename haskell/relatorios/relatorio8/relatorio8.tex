\documentclass{article}
\usepackage[utf8]{inputenc}
\usepackage[a4paper, left=20mm, right=20mm, top=20mm, bottom=20mm]{geometry} % Formatação da página

\begin{document}
    \begin{center}
        \section*{INE5416 - Paradigmas da Programação (2015/2)}
        \textbf{\textit{Relatório 8: Listas e Array} \\
        Caique Rodrigues Marques 13204303}
    \end{center}
    
    \section*{Parte 1}
        \begin{itemize}
            \item Na linguagem de programação C, os arrays podem ser declarados usando colchetes [], como por exemplo \texttt{int array[]}, ou usando a notação de ponteiros, como \texttt{int *array}. Por conta da tipagem, C não permite que um array contenha elementos de diferentes tipos, portanto, é inviável algo como \texttt{\{1, 2, 3, 'a', '0'\}}. A inicialização dos arrays pode ser de forma dinâmica (com o uso de \texttt{malloc}, \texttt{calloc} ou \texttt{alloca}) ou com o uso de chaves compondo os elementos presentes no array, como por exemplo \texttt{\{1,2,3\}}. O conceito de listas pode variar dependendo da linguagem, em C não há implementações nativas quanto a listas, mas há suporte para estruturas de dados como listas ligadas e listas circulares.
            
            \item Em Python há uma alocação mais dinâmica de listas, que pode ser declarado simplesmente como \texttt{list = []} e, como não há preocupação com tipagens na linguagem, pode existir listas com elementos de diferentes tipos, como \texttt{[1, 2, 'a', 'c']}. Inclusive, em Python é possível realizar operações dentro de colchetes para listas, algo como \texttt{[i+1 for i in range(10)]}, onde gerará uma lista com elementos de $1$ a $10$, isto é chamado de \textit{list comprehension} (baseado na implementação existente em Haskell que é semelhante à notação de definição de conjuntos na matemática). Arrays em Python é um pouco diferente, sendo que ela é limitada quanto à tipagem de seus elementos.
            
            \item Em C, arrays passados como parâmetros apontam para a posição de memória onde tal elemento está. Em Python, os elementos são modificados quando passados por parâmetros, para evitar inconsistências, cópias são criadas.
        \end{itemize}
        
    \section*{Parte 2}
        \begin{itemize}
            \item \texttt{relatorio8.c} \\
            Foi usada a estratégia de VLA (\textit{variable-length array}) para evitar o uso de alocação dinâmica, outra facilidade foi a especificação de um macro para defnir uma função de tamanho. \\
            \textbf{Nota:} Para compilação, é necessário especificar biblioteca math: \texttt{gcc relatorio8.c -lm -std=c11}
            
            \item \texttt{relatorio8.py} \\
            Nota-se o aumento da complexidade espacial, em compensação é mais fácil de manipular as estruturas. Foi usada uma alocação estática para as operações.
        \end{itemize}
\end{document}
