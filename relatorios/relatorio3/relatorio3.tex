\documentclass{article}

\usepackage[utf8]{inputenc}
\usepackage{indentfirst}
\usepackage{hyperref}
\usepackage{textcomp}
\usepackage[a4paper, left=20mm, right=20mm, top=20mm, bottom=20mm]{geometry}

\begin{document}
\begin{center}
    \section*{INE5416 - Paradigmas da Programação (2015/2)}
    \textbf{\textit{Relatório 3: Estrutura das Linguagens} \\
    Caique Rodrigues Marques 13204303}
\end{center}

A linguagem Python é uma linguagem interpretada, mas com algumas características bem interessantes. Primeiro, porque tem interface com muitas linguagens como C e Haskell e, em segundo lugar, porque suporta o paradigma procedimental e o paradigma procedural.

Neste relatório há a apresentação de três programas escritos na linguagem Python: bitcoin.py que possui os paradigmas funcional e procedural, apod.py que possui o paradigma procedural e eratosthenes.py que possui o paradigma funcional.

\section*{bitcoin.py}
O programa bitcoin.py foi descrito apenas para fornecer o preço de um Bitcoin nas mais variadas moedas disponíveis pelo mundo. A Bitcoin Price Index API, que foi usada para fornecer os valores, é provinda pelo \href{http://www.coindesk.com/api/}{CoinDesk}. Mais informações podem ser encontradas no link.

Rapidamente apresentando, o Bitcoin é uma criptomoeda baseada na rede \textit{peer-to-peer} (P2P) sem interferência de terceiros ou governos, é totalmente open-source, assim sendo, seu valor em dinheiro é instável e sempre varia. Isto é o que define o paradigma procedural do programa, pois, a cada momento em que ele for executado, sempre retornará um valor diferente, independete da moeda escolhida.

Há três funções que definem o programa: help(), get\_country() e default(), além do main, onde acontece a execução do programa:
\begin{itemize}
    \item A função default() apenas retorna o preço de um Bitcoin nas três moedas definidas na API padrão do Bitcoin Price Index: o Dólar estadunidense, a Libra inglesa e o Euro.
\begin{verbatim}caique@marvin:~$ python3 bitcoin.py 
Bitcoin price in Sep 13, 2015 20:08:00 UTC
[USD] $229.3037 - United States Dollar
[GBP] £148.6140 - British Pound Sterling
[EUR] €202.2431 - Euro
\end{verbatim}
    
    \item A função get\_country() retorna o preço do Bitcoin em outras moedas definidas como argumento na compilação do programa, incluindo as três usadas no padrão default.
\begin{verbatim}caique@marvin:~$ python3 bitcoin.py brl jpy cny
Bitcoin price in Sep 13, 2015 20:22:00 UTC
[BRL] $888.6172 - Brazilian Real
[JPY] $27,664.6011 - Japanese Yen
[CNY] $1,473.3050 - Chinese Yuan
    \end{verbatim}
    
    \item A função help() retorna a lista de moedas disponíveis onde o preço de um Bitcoin pode ser visto. Esta função é o que define o paradigma funcional do programa, pois, usando o argumento "help" na compilação do programa, ela retornará a lista com as moedas e os respectivos países que a adotam.
\begin{verbatim}caique@marvin:~$ python3 bitcoin.py help
AED - United Arab Emirates Dirham
AFN - Afghan Afghani
ALL - Albanian Lek
AMD - Armenian Dram
ANG - Netherlands Antillean Guilder
AOA - Angolan Kwanza
ARS - Argentine Peso
AUD - Australian Dollar    
\end{verbatim}
\end{itemize}

\newpage
\section*{apod.py}
O programa apod.py é baseado na iniciativa \href{http://apod.nasa.gov/apod/astropix.html}{Astronomy Picture of the Day} da NASA, onde cada dia é mostrada uma imagem retirada por um astrônomo e ele a descreve. A API usada para o programa pode ser encontrada no \href{https://data.nasa.gov/developer}{Portal de Dados da NASA}.

\sloppy
O programa tem a mesma funcionalidade da ferramenta provinda pela agência espacial, mostrar uma imagem de astronomia a cada dia. Isto faz com que o programa seja totalmente procedural, pois, a cada entrada num dia, pode retornar uma resposta diferente e, inclusive, o algoritmo foi montado de forma em que ele siga uma sequência de passos (procedures), sem o uso de funções.

\begin{verbatim}
caique@marvin:~$ python3 apod.py 

ASTRONOMY PICTURE OF THE DAY
A Partial Solar Eclipse over Texas
http://apod.nasa.gov/apod/image/1509/TexasEclipse_Westlake_1080.jpg
It was a typical Texas sunset except that most of the Sun was missing. The location of the
missing piece of the Sun was not a mystery -- it was behind the Moon. Featured here is one 
of the more interesting images taken of a partial solar eclipse that occurred in 2012, 
capturing a temporarily crescent Sun setting in a reddened sky behind brush and a windmill.
The image was taken about 20 miles west of Sundown, Texas, USA, just after the ring of fire 
effect was broken by the Moon moving away from the center of the Sun. Today a new partial 
solar eclipse of the Sun will be visible from Earth. Unfortunately for people who live in
Texas, today's eclipse can only be seen from southern Africa and Antarctica.
\end{verbatim}

\section*{eratosthenes.py}
O programa é um algoritmo simples para achar todos os primos até um dado número máximo. Para determinar se um número é primo, divide-o por todos os números a partir de dois até a sua raiz quadrada, se esse número não for divisível por todos os números do intervalo, ele é primo.

Esta técnica é chamada de Crivo de Eratóstenes e foi criada pelo matemático grego Eratóstenes de Cirene, que viveu entre 285-194 a.C.. O programa possui o paradigma procedural porque, sempre que um número máximo qualquer for especificado, ele sempre vai retornar a mesma sequência referente ao número.
\sloppy
\begin{verbatim}caique@marvin:~$ python3 eratosthenes.py 1000

2 3 5 7 11 13 17 19 23 29 31 37 41 43 47 53 59 61 67 71 73 79 83 89 97 101 103 107 109 
113127 131 137 139 149 151 157 163 167 173 179 181 191 193 197 199 211 223 227 229 233 
239 241 251 257 263 269 271 277 281 283 293 307 311 313 317 331 337 347 349 353 359 367 
373 379 383 389 397 401 409 419 421 431 433 439 443 449 457 461 463 467 479 487 491 499
503 509 521 523 541 547 557 563 569 571 577 587 593 599 601 607 613 617 619 631 641 643
647 653 659 661 673 677 683 691 701 709 719 727 733 739 743 751 757 761 769 773 787 797
809 811 821 823 827 829 839 853 857 859 863 877 881 883 887 907 911 919 929 937 941 947 
953 967 971 977 983 991 997 
\end{verbatim}
\end{document}
