\documentclass{article}

\usepackage[utf8]{inputenc}
\usepackage{indentfirst}
\usepackage{hyperref}
\usepackage{textcomp}
\usepackage[a4paper, left=20mm, right=20mm, top=20mm, bottom=20mm]{geometry}

\begin{document}
\begin{center}
    \section*{INE5416 - Paradigmas da Programação (2015/2)}
    \textbf{\textit{Relatório 3: Estrutura das Linguagens} \\
    Caique Rodrigues Marques 13204303}
\end{center}

A linguagem Python é uma linguagem interpretada, mas com algumas características bem interessantes. Primeiro, porque tem interface com muitas linguagens como C e Haskell e, em segundo lugar, porque suporta o paradigma procedimental e o paradigma procedural.

Neste relatório há a apresentação de dois programas escritos na linguagem Python: bitcoin.py que possui os paradigmas funcional e procedural e apod.py que possui o paradigma procedural.

\section*{bitcoin.py}
O programa bitcoin.py foi descrito apenas para fornecer o preço de um Bitcoin nas mais variadas moedas disponíveis pelo mundo. A Bitcoin Price Index API, que foi usada para fornecer os valores, é provinda pelo \href{http://www.coindesk.com/api/}{CoinDesk}. Mais informações podem ser encontradas no link.

Rapidamente apresentando, o Bitcoin é uma criptomoeda baseada na rede \textit{peer-to-peer} (P2P) sem interferência de terceiros ou governos, é totalmente open-source, assim sendo, seu valor em dinheiro é instável e sempre varia. Isto é o que define o paradigma procedural do programa, pois, a cada momento em que ele for executado, sempre retornará um valor diferente, independete da moeda escolhida.

Há três funções que definem o programa: help(), get\_country() e default(), além do main, onde acontece a execução do programa:
\begin{itemize}
    \item A função default() apenas retorna o preço de um Bitcoin nas três moedas definidas na API padrão do Bitcoin Price Index: o Dólar estadunidense, a Libra inglesa e o Euro.
\begin{verbatim}caique@marvin:~$ python bitcoin.py 
Bitcoin price in Sep 13, 2015 20:08:00 UTC
[USD] $229.3037 - United States Dollar
[GBP] £148.6140 - British Pound Sterling
[EUR] €202.2431 - Euro
\end{verbatim}
    
    \item A função get\_country() retorna o preço do Bitcoin em outras moedas definidas como argumento na compilação do programa, incluindo as três usadas no padrão default.
\begin{verbatim}caique@marvin:~$ python bitcoin.py brl jpy cny
Bitcoin price in Sep 13, 2015 20:22:00 UTC
[BRL] $888.6172 - Brazilian Real
[JPY] $27,664.6011 - Japanese Yen
[CNY] $1,473.3050 - Chinese Yuan
    \end{verbatim}
    
    \item A função help() retorna a lista de moedas disponíveis onde o preço de um Bitcoin pode ser visto. Esta função é o que define o paradigma funcional do programa, pois, usando o argumento "help" na compilação do programa, ela retornará a lista com as moedas e os respectivos países que a adotam.
\begin{verbatim}caique@marvin:~$ python bitcoin.py help
AED - United Arab Emirates Dirham
AFN - Afghan Afghani
ALL - Albanian Lek
AMD - Armenian Dram
ANG - Netherlands Antillean Guilder
AOA - Angolan Kwanza
ARS - Argentine Peso
AUD - Australian Dollar    
\end{verbatim}
\end{itemize}

\newpage
\section*{apod.py}
O programa apod.py é baseado na iniciativa da NASA, \href{http://apod.nasa.gov/apod/astropix.html}{Astronomy Picture of the Day}, onde cada dia é mostrada uma imagem retirada por um astrônomo e ele a descreve. A API usada para o programa pode ser encontrada no \href{https://data.nasa.gov/developer}{Portal de Dados da NASA}.

\sloppy
O programa tem a mesma funcionalidade da ferramenta provinda pela NASA, mostrar uma imagem de astronomia a cada dia. Isto faz com que o programa seja totalmente procedural, pois, a cada entrada num dia, pode retornar uma resposta diferente e, inclusive, o algoritmo foi montado de forma em que ele siga uma sequência de passos (procedures), sem o uso de funções.

\begin{verbatim}
caique@marvin:~$ python apod.py 

ASTRONOMY PICTURE OF THE DAY
A Partial Solar Eclipse over Texas
http://apod.nasa.gov/apod/image/1509/TexasEclipse_Westlake_1080.jpg
It was a typical Texas sunset except that most of the Sun was missing. The location of the
missing piece of the Sun was not a mystery -- it was behind the Moon. Featured here is one 
of the more interesting images taken of a partial solar eclipse that occurred in 2012, 
capturing a temporarily crescent Sun setting in a reddened sky behind brush and a windmill.
The image was taken about 20 miles west of Sundown, Texas, USA, just after the ring of fire 
effect was broken by the Moon moving away from the center of the Sun. Today a new partial 
solar eclipse of the Sun will be visible from Earth. Unfortunately for people who live in
Texas, today's eclipse can only be seen from southern Africa and Antarctica.
\end{verbatim}
\end{document}
