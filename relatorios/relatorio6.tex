\documentclass{article}
\usepackage[utf8]{inputenc}
\usepackage{listings} % Uso de trechos de código no texto
\usepackage{indentfirst} % indentar primeiro parágrafo (desativado por padrão)
\usepackage[a4paper, left=20mm, right=20mm, top=20mm, bottom=20mm]{geometry} % Formatação da página

\begin{document}
\lstset{language=Haskell} % Definindo o uso de trechos de código Haskell no texto
\lstset{language=C++} % Definindo o uso de trechos de código C++ no texto
\begin{center}
    \section*{INE5416 - Paradigmas da Programação (2015/2)}
    \textbf{\textit{Relatório 6: Classes e Tipos} \\
    Caique Rodrigues Marques 13204303}
\end{center}

\section*{Parte 1}
    Embora Haskell também possa ser orientado a objetos, o conceito de classes em Haskell tem a função de parametrizar o polimorfismo.

    \subsection*{Polimorfismo}
        Nas linguagens de programação, o polimorfismo é definição de uma interface para diferentes arquivos, ou seja, uma interface pode ter várias formas. Um exemplo, dada uma interface genérica e dois arquivos distintos que implementam esta interface, portanto, de um olhar diferente, a interface possui duas formas de operar. Há dois tipos de polimorfismo:
        \begin{itemize}
            \item \textbf{Polimorfismo Paramétrico: } Quando uma função não possui um tipo especificado, então pode ser usado transparentemente por qualquer tipo, ou seja, a função pode retornar sempre um mesmo valor independente do tipo que foi aplicada a ela. Em orientação a objetos isto é conhecido como programação genérica, como, por exemplo, o uso de templates.
            
            \item \textbf{Polimorfismo Ad Hoc:} Ao contrário do paramétrico, o polimorfismo ad hoc se trata de funções que podem se comportar e retornar valores diferentes dependendo do tipo em que foi usado, por exemplo, especificando o uso em um inteiro pode retornar diferente se usando em um char. Ad-hoc é popular em orientação a objetos e também é conhecido como sobrecarregamento de funções (functions overloading) onde as funções têm a mesma assinatura, mas comportamentos diferentes.
        \end{itemize}
        
        A linguagem Haskell, cujo paradigma é o funcional, implementa os dois tipos de polimorfismo. Em C++ e Java, cujos paradigmas são a orientação a objetos e como já dito antes, o polimorfismo ad hoc pode ser implementado para complementar o conceito de herança. A seguir um exemplo de function overloading em C++ (semelhante pode ser aplicado em Java).
        \begin{lstlisting}[language=C++, frame=single]
            #include <string>
            
            int add(int a, int b) 
            {
                return a + b;
            }
            
            std::string add(const char *a, const char *b)
            {
                std::string result(a);
                result += b;
                return result;
            }
        \end{lstlisting}
        
        Note o método add(), ele foi aplicado a dois tipos: inteiros e strings. Ambas as funções têm comportamentos semelhantes, mas em inteiros o valor retornado é a soma de dois inteiros, enquanto em strings, o valor retornado é a concatenação de duas strings.
        
        Em Haskell, o funcionamento de polimorfismo ad hoc é um pouco mais além, ele não se limita apenas a funções, mas também a tipos, inclusive, Haskell não suporta a possibilidade de duas funções com a mesma assinatura como em C++ ou Java, portanto, ele pode mudar o tipo de retorno em uma função. A seguir um exemplo:
        \begin{lstlisting}[language=Haskell, frame=single]
            f a1 a2 = case (typeOf a1) of
                Int  -> a1 + a2
                Bool -> a1 && a2
                _    -> a1
        \end{lstlisting}
        
        \newpage
        C++ e Java têm implementações semelhantes de polimorfismo paramétrico, como exemplificado antes, com o uso de templates, o método sempre retorna a mesma resposta independente do tipo usado. A seguir, um exemplo de uma implementação de um método de uma lista, em C++ (semelhante pode ser aplicado em Java). Como se pode ver, independente do tipo que for especificado em "T" (seja inteiros, char, strings, booleanos), o método sempre irá adicionar um dado ao final da lista de elementos "T".
        \begin{lstlisting}[language=C++, frame=single]
            template <typename T>
            void Lista<T>::adiciona(T obj)
            {
                this->topo += 1;
                if (this->topo >= this->tamanhoMaximo) {
                    throw "Tamanho maximo excedido";
                }
                this->arranjo[this->topo] = obj;
            }
        \end{lstlisting}
        
        Em Haskell, a implementação de polimorfismo paramétrico é bem semelhante às implementações em orientação a objetos. A seguir um exemplo simples, o parâmetro "[a]" pode ser uma lista de qualquer tipo e a função sempre vai retornar o tamanho, que é definido em inteiros. 
        \begin{lstlisting}[language=Haskell, frame=single]
            length :: [a] -> Int
        \end{lstlisting}
        
    \subsection*{Classes}
        O conceito de classes em Haskell é diferente do conceito de classes em C++, neste, é definido para implementar métodos e interfaces, no primeiro, é definido para implementar interfaces genéricas. Segue um exemplo de uma classe em Haskell onde, dado um número "a", ele o retorna na sua forma real.
        \begin{lstlisting}[language=Haskell, frame=single]
            class (Ord a, Num a) => Real a where
                toRational :: a -> Rational
        \end{lstlisting}
        
        Haskell também possui um outro conceito de classes, as Type Classes, é uma interface, mas com algum comportamento mais específico. Há confusões com implementação de type classes e classes em orientação a objetos, mas são bem distintos como se dá para notar. Como Haskell é uma linguagem fortemente tipada, ele usa variáveis de tipagem para definir as classes que são construídas, como são bastantes usadas as variáveis Eq, Ord, etc..
\end{document}
