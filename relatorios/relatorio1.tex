\documentclass{article}
\usepackage[utf8]{inputenc}

%\title{Relatório 1: Panorama histórico}
%\author{Caique Rodrigues Marques}
%\date{Agosto de 2015}

\begin{document}
\begin{center}
    \section*{INE5416 - Paradigmas da Programação (2015/2)}
    \textbf{\textit{Relatório 1: Panorama Histórico} \\
    Caique Rodrigues Marques 13204303}
\end{center}
Este primeiro roteiro propõe ao aluno o conhecimento de alguns conceitos novos apresentados no conteúdo teórico das aulas sobre panorama histórico, sobre linguagens de programação e seus paradigmas.
\subsection*{Questão 1}
Pesquise sobre o termo "lógica combinatória" (ou \textit{combinatory logic} para info adicional) e tente compreender os combinadores \textbf{SKI}. Consegues diferenciar os combinadores de operadores?
        \begin{itemize}
        \item Todas as operações em cálculo lambda são expressos em SKI numa árvore binária, cujas folhas são os símbolos S, K e I (chamados de combinadores).
        \end{itemize}
        
        \begin{itemize}
        \item Combinador é uma função que segue algumas regras de caso geral, mas é comum combinador ser usado para referir a uma função que abstrai um pouco de funcionalidade comum que é a forma mais básica.
        \end{itemize}

\subsection*{Questão 2} 
O que é a tese de Church-Turing? O que é a prova de computabilidade e como foi construída?
        \begin{itemize}
        \item A tese de Church-Turing foi formulada para o décimo dos 23 problemas propostos por David Hilbert em 1900 na conferência do Congresso Internacional de Matemáticos, o \textit{Entscheidungsproblem}. A solução proposta por Alonzo Church e Alan Turing, em 1930, mostravam a premissa de um algoritmo. \par

Ao definir como um algoritmo, uma sequência de passos finitos, as teses mostravam como algo pode ser computável ou por uma Máquina de Turing que a computasse ou por cálculo lambda.
        \end{itemize}

\subsection*{Questão 3}
Faça uma pesquisa sobre as linguagens de programação existentes e os seus paradigmas. \par
        Existem vários paradigmas, seguem alguns:
        \begin{itemize}
        \item \textit{\textbf{Paradigma lógico:}} baseado em lógica formal, cada sentença é construída a partir de fatos e regras de um problema. \\
        Exemplos: PROLOG e ASP.
        \end{itemize}
        
        \begin{itemize}
        \item \textit{\textbf{Paradigma orientado a objeto:}} baseando-se no conceito de 'objetos'. Cada estrutura de dados montada corresponde a um objeto, que contém dados (aributos) e funções (métodos). \\
        Exemplos: Java, C++ e Python.
        \end{itemize}
        
        \begin{itemize}
        \item \textit{\textbf{Paradigma procedural:}} os programas são compostos por procedures, ou seja, métodos ou funções ou rotinas que definem uma sequência de passos computacionais a se seguir. \\
        Exemplos: C, Fortran, GO e BASIC.
        \end{itemize}

\subsection*{Questão 4}
Lei de Moore, a palestra de Feynman, Peter Shor e a fatoração de números inteiros grandes.
        \begin{itemize}
        \item Lei de Moore foi um conceito proposto por Gordon Moore, da Intel, que dizia que a capacidade de processamento dos computadores em geral dobraria a cada dezoito a vinte e quatro meses. Com o passar do tempo a sua ideia foi se realizando com o aumento de transistores e diminuição do tamanho dos chips, porém, se suspeitava se ela iria perpetuar. \par

A lei está ameaçada pela própria Intel que, recentemente, adiou o lançamento de processadores de 10nm e lançando a família Skylake 14nm no final de 2015 (os atuais são os da família Broadway, também de 14nm). Por outro lado, a IBM afirmou estar trabalhando na fabricação de processadores de 7nm.
        \end{itemize}
        
        \begin{itemize}
        \item Em 1959, Richard Feynman palestrou sobre o controle da manipulação da metéria em escala atômica, segundo ele, não existe percalços a construção de elementos pequenos compostos por elementos ainda menores, no limite atômico, dando a ideia da possibilidade da nanotecnologia e computação quântica.
        \end{itemize}
        
        \begin{itemize}
        \item Peter Shor foi responsável pelo algoritmo de Shor, um algoritmo que é capaz de fatorar inteiros exponencialmente mais rápido em um computador quântico do que em um computador clássico, o que levantou a questão da possibilidade de criptografia em computadores quânticos
        \end{itemize}
        
\subsection*{Questão 5}
O que é um qubit?
        \begin{itemize}
        \item Um Qubit é um bit quântico, ele pode ter três estados, zero, um e um intermediário, que é um e zero ao mesmo tempo.
        \end{itemize}

\subsection*{Questão 6}
D-WAVE?
        \begin{itemize}
        \item D-Wave Systems é uma empresa canadense que foca em computação quântica. Em 11 de maio de 2011 a empresa lançou o D-Wave One, descrito como "o primeiro computador quântico comercial", a máquina opera em chips de 128 qubits. O D-Wave One não é um computador de propósito geral (uso pessoal), mas sim para resolução de problemas e otimização, principalmente. Um sucessor, D-Wave Two já foi anunciado. 
        \end{itemize}
\end{document}
