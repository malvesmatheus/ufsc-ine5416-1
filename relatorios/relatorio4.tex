\documentclass{article}
\usepackage[utf8]{inputenc}
\usepackage{indentfirst} % indentar primeiro parágrafo (desativado por padrão)
\usepackage{listings} % Uso de trechos de código no texto
\usepackage{amsfonts} % Uso de fontes para conjuntos como R (reais), Z (inteiros), etc.
\usepackage{mathtools} % Permite o uso do gather, alinhar equações matemáticas sem referenciar "$"
\usepackage[a4paper, left=20mm, right=20mm, top=20mm, bottom=20mm]{geometry} % Formatação da página

\begin{document}
\lstset{language=Python} % Definindo o uso de trechos de código Python no texto
\begin{center}
    \section*{INE5416 - Paradigmas da Programação (2015/2)}
    \textbf{\textit{Relatório 4: Cálculo Lambda} \\
    Caique Rodrigues Marques 13204303}
\end{center}
O cálculo lambda é um formalismo algébrico para representar a abstração de funções e argumentos na lógica matemática descrito por Alonzo Church com base na lógica combinatória de Schönfinkel e na teoria da recursividade de Stephen Kleene. Este relatório contém as informações coletadas através de estudos sugeridos, dentre eles, o artigo original de Church: \textit{An unsolvable problem of elementary number theory}.

\section*{Parte 1}
\subsection*{Definições}
O cálculo lambda foi desenvolvido por Alonzo Church como solução para o problema da computabilidade proposto por David Hilbert em 1928 que, assim como o artigo de Alan Turing, define as bases para a nova área da matemática, a Ciência da Computação. A solução de Church foi formulada para dar bases formais ao conceito de computação efetiva. O cálculo lambda é formado por uma expressão definida recursivamente.

Uma expressão lambda é composta por:
\begin{itemize}
    \item Variáveis $x_{1}$, $x_{2}$, ..., $x_{n}$, ...
    \item Os símbolos de abstração lambda $("\lambda")$ e ponto (".")
    \item Parênteses ()
\end{itemize}
O conjunto de expressões lambda, A, pode ser definida indutivamente:
\begin{enumerate}
    \item Se x é uma variável, então x$\in$A
    \item Se x é uma variável e M$\in$A, então ($\lambda$x.M)$\in$A
    \item Se M, N$\in$ A, então (M, N)$\in$A
\end{enumerate}
Instâncias da segunda regra são chamadas de abstrações e instâncias da terceira regra são chamadas de aplicações. Segue alguns exemplos:
\begin{itemize}
    \item $\lambda x.x^{2}+xy+z$ $(3)$ = $9+3y+z$
    \item $\lambda x.x$
    \item $\lambda t.t + 2(\lambda y.y^{2} + 2y + 1 (3))$ = $\lambda t.t + 2$ $(9+6+1)$  = $\lambda t.t + 2$ $(16)$ = $18$
    \item $\lambda y.\sqrt[2]{y}$ $(t)$ = $\lambda t.\sqrt[2]{t}$
\end{itemize}
Uma \textbf{abstração} é definida de forma que, usando o primeiro exemplo, $\lambda x$ corresponde ao lambda-termo e $x^{2}+xy+z$ corresponde à expressão. O $(3)$ corresponde à \textbf{aplicação} na expressão lambda, portanto, a aplicação do número três no lambda-termo, tal qual uma função matemática, onde dada uma função $f(x)$ qualquer, um valor qualquer presente no domínio de $f(x)$ pode ser aplicado em x. O cálculo lambda é a base das linguagens de programação com o paradigma funcional. A seguir, um exemplo em Python, que é uma linguagem multiparadigma. Nota-se que as funções anônimas do Python segue o mesma sintaxe e semântica provinda das definições do cálculo lambda.
    \begin{lstlisting}
    def sqrt(n):
        return lambda n: n*n
    \end{lstlisting}
A implementação acima, é o mesmo que $\lambda n. n^{2}$ $(n)$, onde "n" corresponde a um número qualquer. Nota-se que o valor de $\lambda n$ corresponde à \textbf{variável} considerada na questão, enquanto $(n)$ corresponde ao \textbf{argumento} que será usado na função, mesmo nome usado em linguagens ao definir a assinatura de uma função (assinatura corresponde a "def sqrt(n)", onde (n) é o argumento).

\newpage
\subsection*{Reduções}
Expressões em cálculo lambda também podem sofrer mudanças de comportamento, isto é chamado de redução e há três tipos:
    \begin{itemize}
        \item $\alpha-$conversão, onde se altera o limite das variáveis. Geralmente pode ser usada para alterar o nome da variável, embora as regras precisas não sejam triviais. \\
        Exemplo: $\lambda x.x$ $(y)$
        \item $\beta-$redução, onde se aplica funções nos argumentos de uma expressão lambda. \\ 
        Exemplo: $\lambda y.y +2$ $(3x+2y+7z)$
        \item $\eta-$conversão, significa que, dada duas funções, elas só serão iguais se, e somente se, elas dão o mesmo resultado para todos os argumentos definidos. \\
        Exemplo: $\lambda x.(f x)$
    \end{itemize}
    
\section*{Parte 2}
Como dito anteriormente, o cálculo lambda é a base das linguagens de programação funcional. Um exemplo a citar é Haskell, onde o $\lambda$ é substituído por "\textbackslash" e o ponto "." por "$->$"
\begin{enumerate}
    \item Realizando uma operação modular básica, é possível remover o primeiro elemento divisível por três na lista.
    \begin{verbatim}
    Prelude> List.deleteBy(\x y->y `mod` x == 0)3[5..10]
    [5,7,8,9,10]
    \end{verbatim}
    \item De forma semelhante, para filtrar os elementos não divisíveis por quatro na lista.
    \begin{verbatim}
    Prelude> List.filter(\x ->x `mod` 4 == 0)[4..19]
    [4,8,12,16]
    \end{verbatim}
    \item O valor da expressão:
    \begin{verbatim}
    Prelude> [ x | x <- [1..4], y <- [x..5], (x+y) `mod` 2 == 0 ]
    [1,1,1,2,2,3,3,4]
    \end{verbatim}
\end{enumerate}

\section*{Lista de exercícios 1: Funções}
\begin{enumerate}
    \item Inversa de $f(x) = \frac{x+1}{x^{2}}$
        \begin{gather*}
            f(x) = y = \frac{x+1}{x^{2}} \\
            yx^{2} = x+1 \\
            yx^{2}-x-1 = 0 \\
            - \\
            \delta = 1^{2} + 4y1 \\
            \delta = 1 + 4y \\
            - \\
            \mathbf{f^{-1}(x) = \frac{1\pm \sqrt{1+4y}}{2y}}
        \end{gather*}
    \item $f(x-1) = (x^{2}-1)$, $f(x)?$
        \begin{gather*}
            x = x+1: \\
            f((x+1)-1) = ((x+1)^{2}-1) \\
            \mathbf{f(x) = x^{2} + 2x}
        \end{gather*}
    \item Seja $f(x) = x+\frac{1}{x}$, prove que $(f(x))^{3} = f(x^{3})+3f\left(\frac{1}{x}\right).$
        \begin{gather*}
            f(x^{3}) = x^{3}+\frac{1}{x^{3}} = \frac{x^{6}+1}{x^{3}} \\
            f\left(\frac{1}{x}\right) = \frac{1}{x}+x = f(x) \\
            ----- \\
            f(x) = x+\frac{1}{x} \\
            (f(x))^{3} = \left(x+\frac{1}{x}\right)^{3} \\
            (f(x))^{3} = \left(x^{2}+2+\frac{1}{x^{2}}\right).\left(x+\frac{1}{x}\right) \\
            (f(x))^{3} = x^{3}+2x+\frac{x}{x}+\frac{x^{2}}{x}+\frac{2}{x}+\frac{1}{x^{3}} \\
            (f(x))^{3} = x^{3}+\frac{1}{x^{3}}+2x+\frac{1}{x}+x+\frac{2}{x} \\
            (f(x))^{3} = x^{3}+\frac{1}{x^{3}}+3x+\frac{3}{x} \\
            (f(x))^{3} = \left(x^{3}+\frac{1}{x^{3}}\right)+3\left(x+\frac{1}{x}\right) \\
            \mathbf{(f(x))^{3} = f(x^{3})+3f\left(\frac{1}{x}\right)}
        \end{gather*}
    \item Seja $f(x) = \frac{|a|}{a}, x \neq 0$ ache $|f(a)-f(-a)|$.
        \begin{gather*}
            f(a) = \frac{|a|}{a} = \frac{a}{a} = 1 \\
            f(-a) = \frac{|-a|}{-a} = \frac{a}{-a} = -1 \\
            ----\\
            |f(a)-f(-a)| = |1-(-1)| = |2| = 2
        \end{gather*}
    \item Dadas $f(x) = x^{4}$, $g(x) = \sqrt{1+x^{3}}$, $h(x) = \frac{x^{2}+1}{2x+1}$, todos cujo domínio é $\mathbb{R} \to [0, \infty) $, ache cada composição e seus respectivos domínios e contradomínios:
        \begin{enumerate}
        \item[a.] $f \circ g(x)$
            \begin{gather*}
                f \circ g(x) = f(g(x)) \\
                f(g(x)) = f(\sqrt{1+x^{3}}) \\
                f(\sqrt{1+x^{3}}) = (\sqrt{1+x^{3}})^{4} \\
                (\sqrt{1+x^{3}})^{4} = \mathbf{1+2x^{3}+x^{6}} \\
                \text{Domínio} = \mathbb{R} \\
                \text{Contradomínio} = [0, \infty)
            \end{gather*}
        \end{enumerate}
        \item[b.] $f \circ g \circ h(x)$
            \begin{gather*}
                f \circ g \circ h(x) = f(g(h(x))) \\
                f(g(h(x))) = f\left(g\left(\frac{x^{2}+1}{2x+1}\right)\right) \\
                f\left(g\left(\frac{x^{2}+1}{2x+1}\right)\right) = f\left(\frac{(\sqrt{1+x^{3}})^{2}+1}{2.(\sqrt{1+x^{3}})+1}\right) \\
                f\left(\frac{(\sqrt{1+x^{3}})^{2}+1}{2.(\sqrt{1+x^{3}})+1}\right) = \frac{(\sqrt{1+x^{12}})^{2}+1}{2.(\sqrt{1+x^{12}})+1} \\
                \frac{(\sqrt{1+x^{12}})^{2}+1}{2.(\sqrt{1+x^{12}})+1} = \frac{1+x^{12}+1}{2.(\sqrt{1+x^{12}}+1)} \\
                \frac{1+x^{12}+1}{2.(\sqrt{1+x^{12}}+1)} = \mathbf{\frac{2+x^{12}}{2.(\sqrt{1+x^{2}})+1}} \\
                \text{Domínio} = \mathbb{R} \\
                \text{Contradomínio} = [0, \infty)
            \end{gather*}
\end{enumerate}

\section*{Lista de exercícios 2: Cálculo Lambda}
\begin{enumerate}
    \item Transforme as funções em expressões lambda. \\
    (a) $f(x)=x^{2}+4 : \mathbf{\lambda x.x^{2}+4 (x)}$ \\
    (b) $f(x)=\sum{x=1}^{x=10}x : \mathbf{\lambda x.x[1...10] (x)}$ \\
    (c) $f(a, b)=a+b : \mathbf{\lambda ab.a+b (a)(b)}$ \\
    (d) $f(x) = x.x^{-1} : \mathbf{\lambda x.1 (x)}$ \\
    
    \item Calcule as expressões lambda. \\
    (a) $\lambda x.(\lambda y.y^{2} - (\lambda z.(z+x)4)3)2$ \\
        $\lambda x.(\lambda y.y^{2} - (4+x)3)2$ \\
        $\lambda x.(3^{2} - (4+x))2$ \\
        $9 - 4 - 2 = \mathbf{3}$ \\ \\
    (b) $\lambda x.x + (\lambda y.y^{2}(b))(a)$ \\
        $\lambda x.x + (b^{2})(a)$ \\
        $a + (b^{2}) = \mathbf{a+b^{2}}$ \\ \\
    (c) $\lambda x.(\lambda y.(x+(\lambda x.8)-y)6)5$ \\
        $\lambda x.(\lambda y.(x+8-y)6)5$ \\
        $\lambda x.(x+8-6)5$ \\
        $5+8-6 = \mathbf{7}$ \\ \\
    (d) $\lambda xy.x+y (3)(7)$ \\
        $3+7 = \mathbf{10}$
    
    \item Reduzir as expressões na forma normal, quando possível. \\
    (a) $\lambda x.x(xy)(\lambda u.u)$ \\
        $\lambda u.u(\lambda u.uy)$ = $\mathbf{\lambda u.uy}$ \\ \\
    (b) $\lambda y.(\lambda x.y^{2}+x)(z)$ = $\mathbf{\lambda x.z^{2}+x}$ \\ \\
    (c) $\lambda x.(\lambda y.(yx)\lambda i.i)\lambda p.\lambda q.p$ \\
        $\lambda x. (\lambda i.ix)\lambda p.\lambda q.p$ \\
        $\lambda i.i(\lambda p.\lambda q.p)$  = $\mathbf{\lambda p.\lambda q.p}$ \\ \\
    (d) $\lambda x.x(\lambda y.(\lambda x.xy)x)$ \\
        $\lambda x.x(\lambda x.xx)$ = $\mathbf{\lambda x.xx}$ \\ \\
    (e) $(\lambda x.xx)(\lambda y.y)$  = $\lambda y.y(\lambda y.y)$ = $\mathbf{\lambda y.y}$
\end{enumerate}
\end{document}
