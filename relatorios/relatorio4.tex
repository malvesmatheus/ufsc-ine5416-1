\documentclass{article}
\usepackage[utf8]{inputenc}
\usepackage{indentfirst}
\usepackage{listings}
\usepackage[a4paper, left=20mm, right=20mm, top=20mm, bottom=20mm]{geometry}

\begin{document}
\lstset{language=Python}
\begin{center}
    \section*{INE5416 - Paradigmas da Programação (2015/2)}
    \textbf{\textit{Relatório 4: Cálculo Lambda - Parte 1} \\
    Caique Rodrigues Marques 13204303}
\end{center}
O cálculo lambda é um formalismo algébrico para representar a abstração de funções e argumentos na lógica matemática descrito por Alonzo Church com base na lógica combinatória de Schönfinkel e na teoria da recursividade de Stephen Kleene. Este relatório contém as informações coletadas através de estudos sugeridos, dentre eles, o artigo original de Church: \textit{An unsolvable problem of elementary number theory}.

\section*{Definições}
O cálculo lambda foi desenvolvido por Alonzo Church como solução para o problema da computabilidade proposto por David Hilbert em 1928 que, assim como o artigo de Alan Turing, define as bases para a nova área da matemática, a Ciência da Computação. A solução de Church foi formulada para dar bases formais ao conceito de computação efetiva. O cálculo lambda é formado por uma expressão definida recursivamente.

Uma expressão lambda é composta por:
\begin{itemize}
    \item Variáveis $x_{1}$, $x_{2}$, ..., $x_{n}$, ...
    \item Os símbolos de abstração lambda $("\lambda")$ e ponto (".")
    \item Parênteses ()
\end{itemize}
O conjunto de expressões lambda, A, pode ser definida indutivamente:
\begin{enumerate}
    \item Se x é uma variável, então x$\in$A
    \item Se x é uma variável e M$\in$A, então ($\lambda$x.M)$\in$A
    \item Se M, N$\in$ A, então (M, N)$\in$A
\end{enumerate}
Instâncias da segunda regra são chamadas de abstrações e instâncias da terceira regra são chamadas de aplicações. Segue alguns exemplos:
\begin{itemize}
    \item $\lambda x.x^{2}+xy+z$ $(3)$ = $9+3y+z$
    \item $\lambda x.x$
    \item $\lambda t.t + 2(\lambda y.y^{2} + 2y + 1 (3))$ = $\lambda t.t + 2$ $(9+6+1)$  = $\lambda t.t + 2$ $(16)$ = $18$
    \item $\lambda y.\sqrt[2]{y}$ $(t)$ = $\lambda t.\sqrt[2]{t}$
\end{itemize}
Uma \textbf{abstração} é definida de forma que, usando o primeiro exemplo, $\lambda x$ corresponde ao lambda-termo e $x^{2}+xy+z$ corresponde à expressão. O $(3)$ corresponde à \textbf{aplicação} na expressão lambda, portanto, a aplicação do número três no lambda-termo, tal qual uma função matemática, onde dada uma função $f(x)$ qualquer, um valor qualquer presente no domínio de $f(x)$ pode ser aplicado em x. O cálculo lambda é a base das linguagens de programação com o paradigma funcional. A seguir, um exemplo em Python, que é uma linguagem multiparadigma. Nota-se que as funções anônimas do Python segue o mesma sintaxe e semântica provinda das definições do cálculo lambda.
    \begin{lstlisting}
    def sqrt(n):
        return lambda n: n*n
    \end{lstlisting}
A implementação acima, é o mesmo que $\lambda n. n^{2}$ $(n)$, onde "n" corresponde a um número qualquer. Nota-se que o valor de $\lambda n$ corresponde à \textbf{variável} considerada na questão, enquanto $(n)$ corresponde ao \textbf{argumento} que será usado na função, mesmo nome usado em linguagens ao definir a assinatura de uma função (assinatura corresponde a "def sqrt(n)", onde (n) é o argumento).

\newpage
\section*{Reduções}
Expressões em cálculo lambda também podem sofrer mundaças de comportamento, isto é chamado de redução e há três tipos:
\begin{itemize}
    \item $\alpha-$conversão, onde se altera o limite das variáveis. Geralmente pode ser usada para alterar o nome da variável, embora as regras precisas não sejam triviais. \\
    Exemplo: $\lambda x.x$ $(y)$
    \item $\beta-$redução, onde se aplica funções nos argumentos de uma expressão lambda. \\ 
    Exemplo: $\lambda y.y +2$ $(3x+2y+7z)$
    \item $\eta-$conversão, onde captura a noção de extensionalidade. Isto siginifica que, dada duas funções, elas só serão iguais se, e somente se, elas dão o mesmo resultado para todos os argumentos definidos. \\
    Exemplo: $\lambda x.(f x)$
\end{itemize}
\end{document}
